\documentclass[]{article}
\usepackage[T1]{fontenc}
\usepackage{lmodern}
\usepackage{amssymb,amsmath}
\usepackage{ifxetex,ifluatex}
\usepackage{fixltx2e} % provides \textsubscript
% use upquote if available, for straight quotes in verbatim environments
\IfFileExists{upquote.sty}{\usepackage{upquote}}{}
\ifnum 0\ifxetex 1\fi\ifluatex 1\fi=0 % if pdftex
  \usepackage[utf8]{inputenc}
\else % if luatex or xelatex
  \ifxetex
    \usepackage{mathspec}
    \usepackage{xltxtra,xunicode}
  \else
    \usepackage{fontspec}
  \fi
  \defaultfontfeatures{Mapping=tex-text,Scale=MatchLowercase}
  \newcommand{\euro}{€}
\fi
% use microtype if available
\IfFileExists{microtype.sty}{\usepackage{microtype}}{}
\ifxetex
  \usepackage[setpagesize=false, % page size defined by xetex
              unicode=false, % unicode breaks when used with xetex
              xetex]{hyperref}
\else
  \usepackage[unicode=true]{hyperref}
\fi
\hypersetup{breaklinks=true,
            bookmarks=true,
            pdfauthor={Dylan Badin, Kevin Thompson},
            pdftitle={First Responders},
            colorlinks=true,
            citecolor=blue,
            urlcolor=blue,
            linkcolor=magenta,
            pdfborder={0 0 0}}
\urlstyle{same}  % don't use monospace font for urls
\setlength{\parindent}{0pt}
\setlength{\parskip}{6pt plus 2pt minus 1pt}
\setlength{\emergencystretch}{3em}  % prevent overfull lines
\setcounter{secnumdepth}{0}

\title{First Responders}
\author{Dylan Badin, Kevin Thompson}
\date{14-December-2015}

\begin{document}
\maketitle

\section{Introduction}\label{introduction}

\begin{itemize}
\itemsep1pt\parskip0pt\parsep0pt
\item
  The year of 2015 has left a profound impact on the nation. This was
  felt on December 2, 2015, when two Islamic extremists opened fire on a
  Christmas party and left 14 dead, and 22 wounded.
\item
  Victims of the tragedy rely on the work of emergency response
  personnel, such as firefighters and police officers, to provide them
  with the life-saving treatment they need following such devastating
  attacks.
\item
  Unfortunately, active shooter incidents delay the response time
  necessary to first action and effectively save their lives from
  preventable deaths. Few studies have been done to investigate the
  causes of death in such tragedies.
\item
  However, the closest comparable data present to analyze such active
  shooter incidents comes from military examples from the field of
  battle open gunfire.
\item
  The most preventable causes of death experienced by these veterans are
  extremity wound hemorrhage and airway obstruction. TCCC has been shown
  to save lives on the battlefield with basic training in \ldots{}
\item
  Civilian personnel in the public eye are already taught basic first
  aid, CPR, and automated external defribillators (AEDs) but this is
  hardly effective in the event of a traumatic scenario
\item
  Incorporating this training into civilian life, could provide the
  necessary knowledge and save a preventable death
\end{itemize}

\section{Methods}\label{methods}

\begin{itemize}
\itemsep1pt\parskip0pt\parsep0pt
\item
  Participants of the study were obtained as volunteers from the city of
  Westminster
\item
  Laypeople include Nursing grads and undergrads, Teachers, city
  workers, security guards, and students
\item
  Teachers, city workers, security guards and students were randomly
  placed into 2 groups: trained individuals and untrained individuals
\item
  Medically trained personnel include firefighters, who are pre-trained
  in disaster scenario in EMS training
\item
  Trained individuals were given a 2-hour TCCC training overview 6 weeks
  prior to the experiment
\item
  Test participants were pre-screened about their basic knowledge of
  disaster scenario, with questions including ``What is the primary
  cause of death in population ages 1-44?'', ``What do you think is the
  standard response time when 911 is called?'', and ``What is your
  primary concern immediately following a disaster or emergency
  situation''.
\item
  Every group was brought individually into a room with the chief of
  police, who informed them of the situation: At the mall with friends
  when a magnitude 7 earthquake strikes. There will be debris on the
  ground and light will be limited. You are tasked with assessing the
  situation and responding.
\item
  The room was situated to simulate a major earthquake with debris and
  lighting problems with 4 victims: 1 deceased, 1 arterial bleeding, 1
  unconscious but breathing, and 1 healthy individual.
\item
  The participants 1st action times and time to solution were recorded
  by an observer playing a victim's friend.
\item
  They were graded based off prewritten sheets with questions as follows
  :
\item
  Data was recorded per each group. Average times of trained, untrained,
  and professionally trained individuals were compared to both the
  arterial bleeding and airway stations
\end{itemize}

\end{document}
